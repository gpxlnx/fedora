
\documentclass[10pt]{article}
\usepackage{array, xcolor}
\usepackage[margin=3cm]{geometry}
\usepackage{longtable}
\usepackage{makecell}
\usepackage{hyperref}
\hypersetup{
  colorlinks=true,
  urlcolor=blue,
}
\definecolor{lightgray}{gray}{0.8}
\newcolumntype{L}{>{\raggedleft}p{0.14\textwidth}}
\newcolumntype{R}{p{0.8\textwidth}}
\newcommand\VRule{\color{lightgray}\vrule width 0.5pt}

\title{\bfseries\Huge Farzad Sadeghi}
\author{thabogre@gmail.com}
\date{2020/02/05}
\begin{document}
\maketitle
\begin{minipage}[ht]{0.48\textwidth}
  Tehran, Iran\\
  +99 9999 9999999
\end{minipage}

\vspace{5mm}
\hrule
\vspace{5mm}

\section*{Education}
\begin{tabular}{L!{\VRule}R}
  2007--2013&{\bf Bachelor's in Electrical Engineering, Iran University of Sceince and Technology, Iran.}\\[5pt]
\end{tabular}

\vspace{5mm}
\hrule
\vspace{5mm}

\section*{Languages}
\begin{tabular}{L!{\VRule}R}
  {\bf English}&{\bf Native-like}\\
  {\bf Persian}&{\bf Native}\\
  {\bf Azerbaijani}&{\bf Native}\\
  German&B2 but without examinatioin\\
\end{tabular}

\vspace{5mm}
\hrule
\vspace{5mm}

\section*{Professional Experience}
\begin{longtable}{L!{\VRule}R}
  {\bf Feb 2019--Now}&{\bf Maharan International}\\
  &RA, Designer and Developer for DMI in our implementation of ERTMS/ETCS onboard, kernel-side.\\[5pt]
  & \\
  & \\
  {\bf August 2016--Now}&{\bf Mutator}\\
  & project mutator is an open-source suite of tools for the C-family languages.\\[5pt]
  & \\
  & \\
  July 2017--November 2017&{\bf Truebit}\\
  &Developing Truebit and the related projects which included writing a WASM interpreter in python. Researching distributed storage networks for Truebit’s needs which right now include IPFS and Filecoin. Implementing a virtual filesystem in C plus a soft floating-point library in Solidity. Also I did most of the CIs for the repos.\\[5pt]
  & \\
  & \\
  September 2016--November 2016& Iranian Electronics Industry Engineering Corp\\
  & Writing an automatic refactoring tool using the Clang front-end in C++. The tool would be the last piece of an automated model-driven development process. The tool was needed since the code generated by e.g. MATLAB wouldn’t be usable as is. The process involved automatic Misra-C:2004 conformation when possible plus injecting NEAT’s proprietary control flow checking mechanisms and more.\\[5pt]
  & \\
  & \\
  May 2016--July 2016&NEAT\\
  &Writing an automatic refactoring tool using the Clang front-end in C++. The tool would be the last piece of an automated model-driven development process. The tool was needed since the code generated by e.g. MATLAB wouldn’t be usable as is. The process involved automatic Misra-C:2004 conformation when possible plus injecting NEAT’s proprietary control flow checking mechanisms and more.\\[5pt]
  & \\
  & \\
  July 2014--March 2016&Crouse\\
  & Embedded software developer, Safety-critical applications: Implementation of the DSP algorithm, testing, automation and tooling. The Software team’s ISO 26262 representative. Helped develop the work-flow for the team by adopting and adapting Code Reviews, FMEA and FTA  for software and tooling alongside some best practices and defensive C programming guidelines.\\[5pt]
  & \\
  & \\
  2010--2012& Safir Language Academy\\
  & English Teacher – Later on specifically IELTS teacher\\[5pt]
\end{longtable}

\vspace{5mm}
\hrule
\vspace{5mm}

\section*{Professional Experience}
\begin{longtable}{L!{\VRule}R}
  {\bf Code generation}& \makecell{A python script that generates binary file format (WASM, ELF, …) parsers in C.\\A python script that generates the C code necessary to add a C structure to Lua as a Lua table.}\\[5pt]
  & \\
  & \\
  {\bf Blockchain}& The main technologies I worked with as a Truebit member is our own protocol, the Ethereum blockchain, IPFS and Filecoin and Web Assembly(WASM). Truebit’s protocol allows for verification of a task in a trustless network (e.g. blockchains). I’m working on a WASM interpreter in python, a virtual filesystem in C and a soft floating-point implementation in Solidity for the Ethereum Virtual Machine as a library.\\[5pt]
  & \\
  & \\
  {\bf C/C++ Tooling}& Working on an experimental suite of tools aimed at developing C/C++ development. The Project includes a mutation tool, a source code obfuscator, and an interactive CLI tool. M0 supports feature extraction from the source code and its AST to find spots that would be better candidates for mutation. Bruiser acts as a C/C++ dissection lab with future features such as allowing you to run compiled functions from inside mutator, plus a JIT to facilitate faster runtimes for killing the mutants. Obfuscator is a C/C++ source obfuscation tool.\\[5pt]
  & \\
  & \\
  {\bf Automatic Refactoring}& Writing an automatic refactoring tool for NEAT srl. as an intern. The purpose of the tool is to implement MISRA-C rules and other safety related measures developed by NEAT on a source code that is generated by a automatic source-code generator tool(MATLAB, …).\\[5pt]
  & \\
  & \\
  {\bf Testing tools for automated testing and code-coverage}& I was in charge of tetsing VectorCAST and Testwell’s CTC and CMT. For the purpose of testing I 	wrote a simple test suite to test the code coverage on both VectorCAST and CTC. The more 	complete version of the suite (which my company did not want for some reason) is on my Github 	page. The VectorCAST test also included me trying to add a non-existing MCU to VectorCAST 	(which was not successful. The harness either crashed or hanged.).\\[5pt]
  & \\
  & \\
  {\bf Defensive programming and Designing safe code}& Studying  defensive programming practices and architectural design techniques to improve code reliability along with being the representative of the software group in the meetings regarding ISO 26262, in charge of studying part 6 of the standard (software design phase). Plus exploring the use of FMEA and FTA.\\[5pt]
  & \\
  & \\
  {\bf AUTOSAR}& Studying AUTOSAR VFB, SWC and RTE documents. I needed to learn about UML and XML and some general OS features to have a better understanding of the standard. AUTOSAR is an open software architecture standard developed for the automotive companies for the automotive companies(OEMs, Tier1 and Tier2).\\[5pt]
  & \\
  & \\
  {\bf ISO 26262}& The task involved studying the standard. I was also the software's group's representative for the meetings held with the hardware and system teams on ISO 26262.\\[5pt]
  & \\
  & \\
  {\bf ERTMS / ETCS}& The task basically invloved turning functional and technical system requirements into functional and technical software requirements, designing and implementing the said requirements.\\[5pt]
  & \\
  & \\
  {\bf Safety related hard Real-time ECU}& Writing the SRS and the SDD for the DSP algorithm, Implementing the DSP algorithm, Implementing the Dataflash read and write module, participating in the meetings to write down the System requirements for the ECU. Wrote a Python script (for Renesas CS+ which provides the python API to CS+) to automate the algorithm's offline unit/integration tests on the MCU. Wrote a python script to count the Function WCETs (worst-case execution time) from the compiled assembly file, the WCET data is read from an XML file. Was in charge of back-to-back/unit/integration/regression testing the algorithm.\\[5pt]
\end{longtable}

\vspace{5mm}
\hrule
\vspace{5mm}

\section*{Programming Languages}
\begin{tabular}{L!{\VRule}R}
  {\bf Programming Languages}&C, C++\\[5pt]
  & \\
  {\bf Scripting Lanugages}&Python, Lua, Bash\\[5pt]
  & \\
  {\bf Other}&AWK, make\\[5pt]
\end{tabular}

\vspace{5mm}
\hrule

\section*{Links}
\begin{tabular}{L!{\VRule}R}
  {\bf \href{https://ir.linkedin.com/in/farzad-sadeghi-08426277}{Linkedin}}& My Linkedin page.\\[5pt]
  {\bf \href {https://github.com/bloodstalker}{Github}}& My Github page.\\[5pt]
  {\bf \href {https://bloodstalker.github.io/mutator/}{mutator}}& mutator\\[5pt]
  {\bf \href{https://github.com/bloodstalker/cgrep}{cgrep}}& grep for C-family source files\\[5pt]
  {\bf faultreiber}&\href{https://github.com/bloodstalker/faultreiber}{faultreiber} generates a binary file format parser in C using an XML description file.\\[5pt]
  {\bf luatablegen}&\href{https://github.com/bloodstalker/luatablegen}{luatablegen} generated C code that adds C structures as Lua tables using an XML description file.\\[5pt]
  {\bf \href{https://github.com/bloodstalker/dwasm}{dwasm}}& A WASM dump script written in python\\[5pt]
  {\bf \href{https://github.com/bloodstalker/delf}{delf}}& an ELF dump script in python\\[5pt]
  {\bf configs and dotfiles}& \href{https://github.com/bloodstalker/fedora}{my dotfles}\\[5pt]
\end{tabular}


\section*{Links}
\begin{longtable}{L!{\VRule}R}
  {\bf Jason Teutsch}& jt@truebit.io\\[5pt]
  {\bf Francesco Zulliani}& francesco.zuliani@neat.it\\[5pt]
  {\bf Hamid Hoseinzadeh}& h.hosseinzade@crouse.ir\\[5pt]
  {\bf Behzad Daemi}& b.daemi@gmail.com\\[5pt]
\end{longtable}

\end{document}

